\documentclass[12pt, fullpage,letterpaper]{article}

\usepackage[margin=1in]{geometry}
\usepackage{url}
\usepackage{amsmath}
\usepackage{amssymb}
\usepackage{xspace}
\usepackage{graphicx}

\newcommand{\semester}{Fall 2021}
\newcommand{\assignmentId}{0}
\newcommand{\releaseDate}{14 Sep, 2021}
\newcommand{\dueDate}{11:59pm, 1 Oct, 2021}

\newcommand{\bx}{{\bf x}}
\newcommand{\bw}{{\bf w}}

\title{CS 5350/6350: Machine Learining \semester}
\author{Project Information}
\date{Handed out: \releaseDate\\
  Due: \dueDate}

\begin{document}
\maketitle


\footnotesize
	\begin{itemize}
		\item You are welcome to talk to other members of the class about
		the homework. I am more concerned that you understand the
		underlying concepts. However, you should write down your own
		solution. Please keep the class collaboration policy in mind.
		
		\item Feel free to discuss the homework with the instructor or the TAs.
		
		\item Your written solutions should be brief and clear. You need to
		show your work, not just the final answer, but you do \emph{not}
		need to write it in gory detail. Your assignment should be {\bf no
			more than 10 pages}. Every extra page will cost a point.
		
		\item Handwritten solutions will not be accepted.
		
		\item The homework is due by \textbf{midnight of the due date}. Please submit
		the homework on \textbf{Canvas}.
		
		\item Some questions are marked {\bf For 6350 students}. Students
		who are registered for CS 6350 should do these questions. Of
		course, if you are registered for CS 5350, you are welcome to do
		the question too, but you will not get any credit for it.
		
	\end{itemize}



\section*{Project Choice (5 points)}

Please select the project type in the below. Note that you can only select one:
\begin{itemize}
	\item Competitive (Kaggle): $\checkmark$
	\item Exploratory
\end{itemize}


\section*{Project Proposal(10 points)} 
If you are doing the Kaggle project, please register at \url{https://www.kaggle.com/t/9a5e3392a78a47ecb04e825588917dbe}  and make a dummy submission. That is, you can generate a random submission following the format (see the example in Canvas announcement), upload it to the platform, and make sure it successfully get scored. Please \textbf{list your Kaggle account name and your UID here} so that we can track your performance. 

\textbf{Kaggle account user name:} ryandalby

\textbf{UID:} u0848407
\\
\\
\noindent If you are doing an exploratory project, please \textbf{write a one-page proposal (do not exceed Page 2)}, and discuss the following:
\begin{itemize}
	\item Who are in the project team (please list the name and UID of your team members, and there are at most \textbf{two} team members).
	\item What problem do you want to address?
	\item Why is it interesting? Why do you want to use machine learing rather than traditional/existing methods?
\end{itemize}






\end{document}
%%% Local Variables:
%%% mode: latex
%%% TeX-master: t
%%% End:
