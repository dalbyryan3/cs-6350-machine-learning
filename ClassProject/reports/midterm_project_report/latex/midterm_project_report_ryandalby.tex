\documentclass[11pt]{article}
\usepackage{euscript}

\usepackage{amsmath}
\usepackage{amsthm}
\usepackage{amssymb}
\usepackage{epsfig}
\usepackage{xspace}
\usepackage{color}
\usepackage{url}
\usepackage[nodayofweek]{datetime}
\usepackage{multirow}
\usepackage{tabularx}
\usepackage{verbatimbox}


%%%%%%%  For drawing trees  %%%%%%%%%
\usepackage{tikz}
\usetikzlibrary{calc, shapes, backgrounds}

%%%%%%%%%%%%%%%%%%%%%%%%%%%%%%%%%
\setlength{\textheight}{9in}
\setlength{\topmargin}{-0.600in}
\setlength{\headheight}{0.2in}
\setlength{\headsep}{0.250in}
\setlength{\footskip}{0.5in}
\flushbottom
\setlength{\textwidth}{6.5in}
\setlength{\oddsidemargin}{0in}
\setlength{\evensidemargin}{0in}
\setlength{\columnsep}{2pc}
\setlength{\parindent}{1em}
%%%%%%%%%%%%%%%%%%%%%%%%%%%%%%%%%
\newdate{date}{26}{10}{2021}
\title{
    Midterm Project Report

    \large{
    CS 6350 Machine Learning
    }  
    
}
\author{
    Ryan Dalby
}
\date{\displaydate{date}}

\begin{document}
\maketitle

\section*{Progress}
\subsection*{Overview}
First I attempted to use SGD logistic regression, but found more consistent convergence results using lbfgs solver. (occasionally did not converge to good optima)
Then plotted learning curve and saw that the model suffers from high bias meaning that we would benefit from a more powerful model. Exploring nonlienar etc.

Then tried SVM linear, got similar results, used a different SVM kernel got results...  



\subsection*{Data Preprocessing}
\subsection*{Logistic Regression}
\subsection*{Support Vector Machine Classification}

\section*{Next Steps}
Try optimizing hyperparameter when using SGD, deep learning, ensemble methods.


% \bibliographystyle{abbrv}
% \bibliography{}

\end{document}
